% This file is based on the template
% beamer/solutions/conference-talks/conference-ornate-20min.de.tex.
% The author Till Tantau <tantau@users.sourceforge.net> granted the
% extra permission to freely use this template as part of this
% package.

\usepackage[english]{babel}

\usepackage[utf8]{inputenc}

\usepackage{hyperref}
\usepackage{multicol}
\usepackage{listings}
\usepackage{color}

\definecolor{mygray}{rgb}{0.5,0.5,0.5}
\lstset{
    language=Perl,
    basicstyle=\small\ttfamily,
    numbers=left,
    numbersep=5pt,
    numberstyle=\tiny\color{mygray},
    aboveskip=0pt,
    belowskip=0pt,
}

\lstdefinestyle{red}{
    basicstyle=\small\ttfamily\color{red}
}

\setlength{\columnseprule}{1pt}
\def\columnseprulecolor{\color{blue}}

%%%%%%%%%%%%%%%%%%%%%%%%%%%%%%%%%%%%%%%%%%%%%%%%%%%%%%%%%%%%%%%%%%%%%%%%%%%%%%%%%%%%%%%%%%%%%%%%%%%

\title{Writing candy webservices with Dancer2::Plugin::CRUD}

\author{David~Zurborg}
\institute[WGG]
{
    WERTGARANTIE Group\\
    Hannover
}

\date[GPW 2015]{German Perl Workshop, Dresden 2015}

%%%%%%%%%%%%%%%%%%%%%%%%%%%%%%%%%%%%%%%%%%%%%%%%%%%%%%%%%%%%%%%%%%%%%%%%%%%%%%%%%%%%%%%%%%%%%%%%%%%

\begin{document}

%%%%%%%%%%%%%%%%%%%%%%%%%%%%%%%%%%%%%%%%%%%%%%%%%%%%%%%%%%%%%%%%%%%%%%%%%%%%%%%%%%%%%%%%%%%%%%%%%%%

\begin{frame}
    \titlepage
    \note[item]{Vorstellung}
    \note[item]{Webservices: Maschinen sprechen mit Maschinen}
    \note[item]{Standards notwendig, müssen implementierbar sein}
    \note[item]{Best Current Practise}
\end{frame}

%%%%%%%%%%%%%%%%%%%%%%%%%%%%%%%%%%%%%%%%%%%%%%%%%%%%%%%%%%%%%%%%%%%%%%%%%%%%%%%%%%%%%%%%%%%%%%%%%%%

\begin{frame}{Table of contents}
    \tableofcontents
\end{frame}

%%%%%%%%%%%%%%%%%%%%%%%%%%%%%%%%%%%%%%%%%%%%%%%%%%%%%%%%%%%%%%%%%%%%%%%%%%%%%%%%%%%%%%%%%%%%%%%%%%%

\section{Introduction}

%%%%%%%%%%%%%%%%%%%%%%%%%%%%%%%%%%%%%%%%%%%%%%%%%%%%%%%%%%%%%%%%%%%%%%%%%%%%%%%%%%%%%%%%%%%%%%%%%%%

\subsection{CRUD and REST}

%%%%%%%%%%%%%%%%%%%%%%%%%%%%%%%%%%%%%%%%%%%%%%%%%%%%%%%%%%%%%%%%%%%%%%%%%%%%%%%%%%%%%%%%%%%%%%%%%%%

\begin{frame}{Database operations}
    \note[item]{Basics for everything}
    \note[item]{Represent methods through all layers}
    \begin{itemize}
    \pause
    \item
        \textbf{C}reate
        \begin{itemize}
            \item Creates a new object
            \item SQL-92: \texttt{INSERT}
        \end{itemize}
    \pause
    \item
        \textbf{R}ead
        \begin{itemize}
            \item Outputs an (one!) object
            \item SQL-92: \texttt{SELECT}
        \end{itemize}
    \pause
    \item
        \textbf{U}pdate
        \begin{itemize}
            \item Edits an object
            \item SQL-92: \texttt{UPDATE}
        \end{itemize}
    \pause
    \item
        \textbf{D}elete
        \begin{itemize}
            \item Removes an object
            \item SQL-92: \texttt{DELETE}
        \end{itemize}
    \end{itemize}
\end{frame}

%%%%%%%%%%%%%%%%%%%%%%%%%%%%%%%%%%%%%%%%%%%%%%%%%%%%%%%%%%%%%%%%%%%%%%%%%%%%%%%%%%%%%%%%%%%%%%%%%%%

\begin{frame}{Representational State Transfer}{A pragmatical programm paradigma}
    \note{Maschinen reden mit Maschinen}
    \begin{itemize}
        \pause
        \item Representation of CRUD operations
        \pause
        \begin{itemize}
            \item Read: \texttt{GET}
            \pause
            \item Create: \texttt{POST}
            \pause
            \item Update: \texttt{PUT}
            \pause
            \item Delete: \texttt{DELETE}
            \pause
            \note[item]{Es gibt auch noch PATCH}
        \end{itemize}
        \item No side effects with \texttt{GET}
        \pause
        \item Unique locations
        \pause
        \item Stateless
        \pause
        \item User-chooseable formats for representation
    \end{itemize}
\end{frame}

%%%%%%%%%%%%%%%%%%%%%%%%%%%%%%%%%%%%%%%%%%%%%%%%%%%%%%%%%%%%%%%%%%%%%%%%%%%%%%%%%%%%%%%%%%%%%%%%%%%

\begin{frame}{Representational State Transfer}{Unique Location Identifier}

\begin{center}

\pause

\large http://www.example.com/customers/33245

\end{center}

\pause
        \begin{itemize}
            \item Scheme: \texttt{http}
            \pause
            \item Host: \texttt{www.example.com}
            \pause
            \item Resource Name: \texttt{customers}
            \pause
            \item Resource ID: \texttt{33245}
        \end{itemize}



\end{frame}

%%%%%%%%%%%%%%%%%%%%%%%%%%%%%%%%%%%%%%%%%%%%%%%%%%%%%%%%%%%%%%%%%%%%%%%%%%%%%%%%%%%%%%%%%%%%%%%%%%%

\begin{frame}{Representational State Transfer}{Status codes}
    \begin{itemize}
        \pause
        \item Wherever possible, use standardized HTTP status codes
        \pause
        \begin{itemize}
            \item Everything OK: \texttt{200 OK}
            \pause
            \item Entity created: \texttt{201 Created}
            \note[item]{Leerer Body, Header enthält \texttt{Location:}-Feld}
            \pause
            \item User error: \texttt{400 Bad Request}
            \note[item]{Für viele andere Fehler gibt es Status-Code}
            \pause
            \item Not found: \texttt{404 Not Found}
            \note[item]{Bezieht sich auf die URI}
            \pause
        \end{itemize}
        \item Every possible error should have a documentation
        \pause
        \item Error message should be readable by machines
        \pause
        \item More information at \url{http://www.restapitutorial.com}\cite{restapitutorial}
    \end{itemize}
\end{frame}

%%%%%%%%%%%%%%%%%%%%%%%%%%%%%%%%%%%%%%%%%%%%%%%%%%%%%%%%%%%%%%%%%%%%%%%%%%%%%%%%%%%%%%%%%%%%%%%%%%%

\subsection{Market analysis}

%%%%%%%%%%%%%%%%%%%%%%%%%%%%%%%%%%%%%%%%%%%%%%%%%%%%%%%%%%%%%%%%%%%%%%%%%%%%%%%%%%%%%%%%%%%%%%%%%%%

\begin{frame}{Overview of current solutions}
    \begin{itemize}
        \pause
        \item CGI.pm
        \pause
        \item Catalyst
        \pause
        \item Raisin
        \pause
        \item Jedi
        \pause
        \item Spiffy
        \pause
        \item Mojolicious
        \pause
        \item Dancer2
        \pause
        \item Dancer2::Plugin::REST
    \end{itemize}
\end{frame}

%%%%%%%%%%%%%%%%%%%%%%%%%%%%%%%%%%%%%%%%%%%%%%%%%%%%%%%%%%%%%%%%%%%%%%%%%%%%%%%%%%%%%%%%%%%%%%%%%%%

\section{Dancer2::Plugin::CRUD}

%%%%%%%%%%%%%%%%%%%%%%%%%%%%%%%%%%%%%%%%%%%%%%%%%%%%%%%%%%%%%%%%%%%%%%%%%%%%%%%%%%%%%%%%%%%%%%%%%%%

\begin{frame}[fragile]{Synopsis}
\pause
\begin{lstlisting}
use Dancer2::Plugin::CRUD;
\end{lstlisting}\pause\begin{lstlisting}[firstnumber=last]
resource('persons',
\end{lstlisting}\pause\begin{lstlisting}[firstnumber=last]
    create => sub {
        my $app = shift;
    }, # POST /persons
\end{lstlisting}\pause\begin{lstlisting}[firstnumber=last]
    read   => sub {
        my ($app, %param) = @_;
        my $persons_id = $param{persons_id};
    }, # GET /persons/1234
\end{lstlisting}\pause\begin{lstlisting}[firstnumber=last]
    update => sub {
        my ($app, %param) = @_;
        my $persons_id = $param{persons_id};
    }, # PUT /persons/1234
\end{lstlisting}\pause\begin{lstlisting}[firstnumber=last]
    delete => sub {
        ...
    }, # DELETE /persons/1234
);
\end{lstlisting}
\end{frame}

%%%%%%%%%%%%%%%%%%%%%%%%%%%%%%%%%%%%%%%%%%%%%%%%%%%%%%%%%%%%%%%%%%%%%%%%%%%%%%%%%%%%%%%%%%%%%%%%%%%

\begin{frame}[fragile]{Read and Index}
\pause
\begin{lstlisting}
resource('persons'
    index => sub {
        my $app = shift;
    }, # GET /persons
\end{lstlisting}\pause\begin{lstlisting}[firstnumber=last]
    read   => sub {
        my ($app, %param) = @_;
        my $persons_id = $param{persons_id};
    }, # GET /persons/1234
);
\end{lstlisting}
\end{frame}

%%%%%%%%%%%%%%%%%%%%%%%%%%%%%%%%%%%%%%%%%%%%%%%%%%%%%%%%%%%%%%%%%%%%%%%%%%%%%%%%%%%%%%%%%%%%%%%%%%%

\begin{frame}[fragile]{Update and Patch}
\pause
\begin{lstlisting}
resource('persons'
    update => sub {
        my ($app, %param) = @_;
        my $persons_id = $param{persons_id};
    }, # PUT /persons/1234
\end{lstlisting}\pause\begin{lstlisting}[firstnumber=last]
    patch => sub {
        my ($app, %param) = @_;
        my $persons_id = $param{persons_id};
    }, # PATCH /persons/1234
);
\end{lstlisting}
\end{frame}

%%%%%%%%%%%%%%%%%%%%%%%%%%%%%%%%%%%%%%%%%%%%%%%%%%%%%%%%%%%%%%%%%%%%%%%%%%%%%%%%%%%%%%%%%%%%%%%%%%%

\begin{frame}[fragile]{Pluralize}
\pause
\begin{itemize}
    \item Philosophical debate
    \pause
    \item Suggestion: every resource in plural
    \pause
    \item There is \texttt{Text::Pluralize} for \textit{Index}
    \note[item]{Und bei der Verkettung}
    \pause
    \item Singular: Create, Read, Update, Patch, Delete, ...
    \pause
    \item Plural: Index
    \pause
    \item[]
\begin{lstlisting}
resource('person(s)'
    index => sub {
        my ($app, %param) = @_;
        my $persons_id = $param{persons_id};
    }, # GET /persons
\end{lstlisting}\pause\begin{lstlisting}[firstnumber=last]
    create => sub {
        my ($app, %param) = @_;
        my $persons_id = $param{persons_id};
    }, # POST /person
);
\end{lstlisting}
\end{itemize}
\end{frame}

%%%%%%%%%%%%%%%%%%%%%%%%%%%%%%%%%%%%%%%%%%%%%%%%%%%%%%%%%%%%%%%%%%%%%%%%%%%%%%%%%%%%%%%%%%%%%%%%%%%

\begin{frame}{Formats}
    \pause
    \begin{itemize}
        \item Content type by extension
        \pause
        \item \texttt{/resource/:id.:format}
        \pause
        \begin{itemize}
            \item YAML: \texttt{/resource/123.yml}
            \pause
            \item JSON: \texttt{/resource/123.json}
            \pause
            \item Dumper: \texttt{/resource/123.dump}
            \pause
            \note[item]{WIP: XML and CBOR}
        \end{itemize}
        \item Input format is also output format
        \pause
        \item from YAML to YAML
        \pause
        \item from JSON to JSON
    \end{itemize}

\end{frame}

%%%%%%%%%%%%%%%%%%%%%%%%%%%%%%%%%%%%%%%%%%%%%%%%%%%%%%%%%%%%%%%%%%%%%%%%%%%%%%%%%%%%%%%%%%%%%%%%%%%

\begin{frame}[fragile]{Chaining}
\pause
\begin{itemize}
\item Chain resources together
\pause
\item[]
\begin{lstlisting}
resource("foo",
    single_id => sub {
\end{lstlisting}\pause\begin{lstlisting}[firstnumber=last]
        resource("bar",
            read => sub {
\end{lstlisting}\pause\begin{lstlisting}[firstnumber=last]
                my ($app, %param) = @_;
                my $foo_id = $param{foo_id};
                my $bar_id = $param{bar_id};
\end{lstlisting}\pause\begin{lstlisting}[firstnumber=last]
            } # GET /foo/123/bar/456
        );
    },
);
\end{lstlisting}
\end{itemize}
\end{frame}

%%%%%%%%%%%%%%%%%%%%%%%%%%%%%%%%%%%%%%%%%%%%%%%%%%%%%%%%%%%%%%%%%%%%%%%%%%%%%%%%%%%%%%%%%%%%%%%%%%%

\begin{frame}[fragile]{Validation}
\pause
\begin{itemize}
\item \texttt{JSON::Schema}\cite{json-schema}
\pause
\item[]
\begin{lstlisting}
resource("persons",
    create => sub :RequestSchema(
\end{lstlisting}\pause\begin{lstlisting}[firstnumber=last,style=red]
        type => 'object',
        properties => {
            firstName => { type => 'string' },
            lastName  => { type => 'string' },
            age => {
                type => 'integer',
                minimum => 0
            }
        }
\end{lstlisting}\pause\begin{lstlisting}[firstnumber=last]
    ) {
        my $app = shift;
    },
);
\end{lstlisting}
\end{itemize}
\end{frame}

%%%%%%%%%%%%%%%%%%%%%%%%%%%%%%%%%%%%%%%%%%%%%%%%%%%%%%%%%%%%%%%%%%%%%%%%%%%%%%%%%%%%%%%%%%%%%%%%%%%

\begin{frame}[fragile]{Documentation}
\pause
\begin{itemize}
\item Make your consumers happy
\pause
\item[]
\begin{lstlisting}
resource("persons",
    index => sub :Description(List all persons in this room) {
        my $app = shift;
    },
);
\end{lstlisting}\pause\begin{lstlisting}[firstnumber=last]

publish_apiblueprint("/doc");
\end{lstlisting}
\end{itemize}
\end{frame}

%%%%%%%%%%%%%%%%%%%%%%%%%%%%%%%%%%%%%%%%%%%%%%%%%%%%%%%%%%%%%%%%%%%%%%%%%%%%%%%%%%%%%%%%%%%%%%%%%%%

\begin{frame}[fragile]{Documentation}
\pause
\begin{itemize}

\item API Blueprint\cite{apiblueprint}
\pause
\item Markdown Syntax
\pause

\item
{\tiny
\begin{multicols}{3}
\begin{semiverbatim}
FORMAT: 1A
# Group persons
## Index [\alert<5>{/persons.\{format\}}]
### Index [GET]
List all persons in this room
+ Parameters
    + format (\alert<6>{required, string})
        + Values
            + `json`
            + `yaml`
            + `dump`
+ Request \alert<7>{JSON} (\alert<8>{application/json})
    + `\alert<9>{/persons.json}`
    + Body
            \alert<10>{\{
               "limit" : 10
            \}}
+ Response 200 (application/json)
    + Body
            \alert<11>{[
               \{
                  "name" : "Alice"
               \},
               \{
                  "name" : "Bob"
               \}
            ]}
\end{semiverbatim}\columnbreak\begin{semiverbatim}
\alert<12>{+ Request YAML (text/x-yaml)
    + `/persons.yml`
    + Body
            --- 
            limit: 10
+ Response 200 (text/x-yaml)
    + Body
            --- 
            - 
              name: Alice
            - 
              name: Bob
}\end{semiverbatim}\columnbreak\begin{semiverbatim}
\alert<13>{+ Request DUMP (text/x-perl)
    + `/persons.dump`
    + Body
            \{
              'limit' => 10
            \}
+ Response 200 (text/x-perl)
    + Body
            [
              \{
                'name' => 'Alice'
              \},
              \{
                'name' => 'Bob'
              \}
            ]
}\end{semiverbatim}
\end{multicols}
}

\pause
\pause
\pause
\pause
\pause
\pause
\pause
\pause
\pause
\pause
Compile blueprint to HTML with aglio\cite{aglio}
\end{itemize}

\end{frame}

%%%%%%%%%%%%%%%%%%%%%%%%%%%%%%%%%%%%%%%%%%%%%%%%%%%%%%%%%%%%%%%%%%%%%%%%%%%%%%%%%%%%%%%%%%%%%%%%%%%

\begin{frame}[fragile]{Unit testing}
\pause
\begin{itemize}
\item Do tests with \texttt{Data::Dumper}
\pause
\item[]
\begin{lstlisting}
use Dancer2::Plugin::CRUD::Test;
my $T = Dancer2::Plugin::CRUD::Test->new('Webservice');
\end{lstlisting}\pause\begin{lstlisting}[firstnumber=last]
my $R = $T->action(
    create => '/echo',
\end{lstlisting}\pause\begin{lstlisting}[firstnumber=last]
    { 'xx' => 'yy' },
\end{lstlisting}\pause\begin{lstlisting}[firstnumber=last]
    expect => 200,
);
\end{lstlisting}\pause\begin{lstlisting}[firstnumber=last]
is_deeply($R->data, { 'xx' => 'yy' } );
\end{lstlisting}
\end{itemize}
\end{frame}

%%%%%%%%%%%%%%%%%%%%%%%%%%%%%%%%%%%%%%%%%%%%%%%%%%%%%%%%%%%%%%%%%%%%%%%%%%%%%%%%%%%%%%%%%%%%%%%%%%%

\appendix

%%%%%%%%%%%%%%%%%%%%%%%%%%%%%%%%%%%%%%%%%%%%%%%%%%%%%%%%%%%%%%%%%%%%%%%%%%%%%%%%%%%%%%%%%%%%%%%%%%%

\section<presentation>*{\appendixname}

%%%%%%%%%%%%%%%%%%%%%%%%%%%%%%%%%%%%%%%%%%%%%%%%%%%%%%%%%%%%%%%%%%%%%%%%%%%%%%%%%%%%%%%%%%%%%%%%%%%

\subsection<presentation>*{Link tips}

%%%%%%%%%%%%%%%%%%%%%%%%%%%%%%%%%%%%%%%%%%%%%%%%%%%%%%%%%%%%%%%%%%%%%%%%%%%%%%%%%%%%%%%%%%%%%%%%%%%

\begin{frame}[allowframebreaks]
  \frametitle<presentation>{Link tips}

\bibliography{talk-gpw2015-de}
\bibliographystyle{unsrt}

\end{frame}

%%%%%%%%%%%%%%%%%%%%%%%%%%%%%%%%%%%%%%%%%%%%%%%%%%%%%%%%%%%%%%%%%%%%%%%%%%%%%%%%%%%%%%%%%%%%%%%%%%%

\begin{frame}[fragile]

\begin{center}

    \huge{
        Follow me at\\
        \url{https://github.com/zurborg}
    }

\end{center}

\end{frame}

%%%%%%%%%%%%%%%%%%%%%%%%%%%%%%%%%%%%%%%%%%%%%%%%%%%%%%%%%%%%%%%%%%%%%%%%%%%%%%%%%%%%%%%%%%%%%%%%%%%

\end{document}

