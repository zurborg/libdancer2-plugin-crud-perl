% This file is based on the template
% beamer/solutions/conference-talks/conference-ornate-20min.de.tex.
% The author Till Tantau <tantau@users.sourceforge.net> granted the
% extra permission to freely use this template as part of this
% package.

\documentclass[aspectratio=1610]{beamer}

\mode<presentation>
{
    \usetheme{Pittsburgh}
    \setbeamercovered{transparent}
}


\usepackage[german]{babel}

\usepackage[utf8]{inputenc}

\usepackage{hyperref}
\usepackage{multicol}
\usepackage{listings}
\usepackage{color}

\definecolor{mygray}{rgb}{0.5,0.5,0.5}
\lstset{
   language=Perl,
    basicstyle=\small\ttfamily,       % the size of the fonts that are used for the code
  numbers=left,                    % where to put the line-numbers; possible values are (none, left, right)
  numbersep=5pt,                   % how far the line-numbers are from the code
  numberstyle=\tiny\color{mygray}, % the style that is used for the line-numbers
  aboveskip=0pt,
  belowskip=0pt,
}

\setlength{\columnseprule}{1pt}
\def\columnseprulecolor{\color{blue}}

%%%%%%%%%%%%%%%%%%%%%%%%%%%%%%%%%%%%%%%%%%%%%%%%%%%%%%%%%%%%%%%%%%%%%%%%%%%%%%%%%%%%%%%%%%%%%%%%%%%

\title{Writing candy webservices with Dancer2::Plugin::CRUD}

%\subtitle{...}

\author{D.~Zurborg}
\institute[PPIT] % (optional, aber oft nötig)
{
    ProPartner.IT GmbH\\
    Osnabrück
}

\date[GPW 2015]{German Perl Workshop, Dresden 2015}

% \pgfdeclareimage[height=0.5cm]{university-logo}{university-logo-filename}
% \logo{\pgfuseimage{university-logo}}

%%%%%%%%%%%%%%%%%%%%%%%%%%%%%%%%%%%%%%%%%%%%%%%%%%%%%%%%%%%%%%%%%%%%%%%%%%%%%%%%%%%%%%%%%%%%%%%%%%%

\begin{document}

%%%%%%%%%%%%%%%%%%%%%%%%%%%%%%%%%%%%%%%%%%%%%%%%%%%%%%%%%%%%%%%%%%%%%%%%%%%%%%%%%%%%%%%%%%%%%%%%%%%

\begin{frame}
  \titlepage
\end{frame}

%%%%%%%%%%%%%%%%%%%%%%%%%%%%%%%%%%%%%%%%%%%%%%%%%%%%%%%%%%%%%%%%%%%%%%%%%%%%%%%%%%%%%%%%%%%%%%%%%%%

\begin{frame}{Was wir zu sehen bekommen}
    \tableofcontents
\end{frame}

%%%%%%%%%%%%%%%%%%%%%%%%%%%%%%%%%%%%%%%%%%%%%%%%%%%%%%%%%%%%%%%%%%%%%%%%%%%%%%%%%%%%%%%%%%%%%%%%%%%

\section{Einführung}

%%%%%%%%%%%%%%%%%%%%%%%%%%%%%%%%%%%%%%%%%%%%%%%%%%%%%%%%%%%%%%%%%%%%%%%%%%%%%%%%%%%%%%%%%%%%%%%%%%%

\subsection{CRUD und REST}

%%%%%%%%%%%%%%%%%%%%%%%%%%%%%%%%%%%%%%%%%%%%%%%%%%%%%%%%%%%%%%%%%%%%%%%%%%%%%%%%%%%%%%%%%%%%%%%%%%%

\begin{frame}{Datenbank-Operationen}
    \begin{itemize}
    \item
        \textbf{C}reate
        \pause
        \begin{itemize}
            \item Anlegen
            \pause
            \item SQL-92: \texttt{INSERT}
        \end{itemize}
    \pause
    \item
        \textbf{R}ead
        \pause
        \begin{itemize}
            \item Ausgeben
            \pause
            \item SQL-92: \texttt{SELECT}
        \end{itemize}
    \pause
    \item
        \textbf{U}pdate
        \pause
        \begin{itemize}
            \item Verändern
            \pause
            \item SQL-92: \texttt{UPDATE}
        \end{itemize}
    \pause
    \item
        \textbf{D}elete
        \pause
        \begin{itemize}
            \item Entfernen
            \pause
            \item SQL-92: \texttt{DELETE}
        \end{itemize}
    \end{itemize}
\end{frame}

%%%%%%%%%%%%%%%%%%%%%%%%%%%%%%%%%%%%%%%%%%%%%%%%%%%%%%%%%%%%%%%%%%%%%%%%%%%%%%%%%%%%%%%%%%%%%%%%%%%

\begin{frame}{Representational State Transfer}{Ein pragmatisches Programmierparadigma}
    \begin{itemize}
        \item Abbilden von CRUD-Operationen
        \pause
        \begin{itemize}
            \item Read: \texttt{GET}
            \pause
            \item Create: \texttt{POST}
            \pause
            \item Update: \texttt{PUT}
            \pause
            \item Delete: \texttt{DELETE}
            \pause
        \end{itemize}
        \item Keine Nebeneffekte bei \texttt{GET}
        \pause
        \item Eindeutigkeit von Adressen
        \pause
        \item Zustandslosigkeit
        \pause
        \item Frei wählbare Repräsentationen
    \end{itemize}
\end{frame}

%%%%%%%%%%%%%%%%%%%%%%%%%%%%%%%%%%%%%%%%%%%%%%%%%%%%%%%%%%%%%%%%%%%%%%%%%%%%%%%%%%%%%%%%%%%%%%%%%%%

\subsection{Marktanalyse}

%%%%%%%%%%%%%%%%%%%%%%%%%%%%%%%%%%%%%%%%%%%%%%%%%%%%%%%%%%%%%%%%%%%%%%%%%%%%%%%%%%%%%%%%%%%%%%%%%%%

\begin{frame}{Überblick bestehender Lösungen}
    \begin{itemize}
        \item CGI
        \pause
        \item Catalyst
        \pause
        \item Raisin
        \pause
        \item Dancer2
        \pause
        \item Dancer2::Plugin::REST
    \end{itemize}
\end{frame}

%%%%%%%%%%%%%%%%%%%%%%%%%%%%%%%%%%%%%%%%%%%%%%%%%%%%%%%%%%%%%%%%%%%%%%%%%%%%%%%%%%%%%%%%%%%%%%%%%%%

\section{Dancer2::Plugin::CRUD}

%%%%%%%%%%%%%%%%%%%%%%%%%%%%%%%%%%%%%%%%%%%%%%%%%%%%%%%%%%%%%%%%%%%%%%%%%%%%%%%%%%%%%%%%%%%%%%%%%%%

\begin{frame}[fragile]{Synopsis}

\begin{lstlisting}
use Dancer2::Plugin::CRUD;
\end{lstlisting}\pause\begin{lstlisting}[firstnumber=last]
resource('person',
\end{lstlisting}\pause\begin{lstlisting}[firstnumber=last]
    create => sub {
        my $app = shift;
    }, # POST /person
\end{lstlisting}\pause\begin{lstlisting}[firstnumber=last]
    read   => sub {
        my ($app, %param) = @_;
        my $person_id = $param{person_id};
    }, # GET /person/1234
\end{lstlisting}\pause\begin{lstlisting}[firstnumber=last]
    update => sub {
        my ($app, %param) = @_;
        my $person_id = $param{person_id};
    }, # PUT /person/1234
\end{lstlisting}\pause\begin{lstlisting}[firstnumber=last]
    delete => sub {
        ...
    }, # DELETE /person/1234
);
\end{lstlisting}
\end{frame}

%%%%%%%%%%%%%%%%%%%%%%%%%%%%%%%%%%%%%%%%%%%%%%%%%%%%%%%%%%%%%%%%%%%%%%%%%%%%%%%%%%%%%%%%%%%%%%%%%%%

\begin{frame}[fragile]{Read and Index}

\begin{lstlisting}
resource('person'
    index => sub {
        my $app = shift;
    }, # GET /person
\end{lstlisting}\pause\begin{lstlisting}[firstnumber=last]
    read   => sub {
        my ($app, %param) = @_;
        my $person_id = $param{person_id};
    }, # GET /person/1234
);
\end{lstlisting}
\end{frame}

%%%%%%%%%%%%%%%%%%%%%%%%%%%%%%%%%%%%%%%%%%%%%%%%%%%%%%%%%%%%%%%%%%%%%%%%%%%%%%%%%%%%%%%%%%%%%%%%%%%

\begin{frame}[fragile]{Update and Patch}

\begin{lstlisting}
resource('person'
    update => sub {
        my ($app, %param) = @_;
        my $person_id = $param{person_id};
    }, # PUT /person/1234
\end{lstlisting}\pause\begin{lstlisting}[firstnumber=last]
    patch => sub {
        my ($app, %param) = @_;
        my $person_id = $param{person_id};
    }, # PATCH /person/1234
);
\end{lstlisting}
\end{frame}

%%%%%%%%%%%%%%%%%%%%%%%%%%%%%%%%%%%%%%%%%%%%%%%%%%%%%%%%%%%%%%%%%%%%%%%%%%%%%%%%%%%%%%%%%%%%%%%%%%%

\begin{frame}[fragile]{Pluralize}

\begin{itemize}
    \item \texttt{Text::Pluralize}
    \pause
    \item Zur Unterscheidung von \textit{Index}
    \pause
    \item Singular: Create, Read, Update, Patch, Delete, ...
    \pause
    \item Plural: Index
    \pause
    \item[]
\begin{lstlisting}
resource('person(s)'
    index => sub {
        my ($app, %param) = @_;
        my $person_id = $param{person_id};
    }, # GET /persons
\end{lstlisting}\pause\begin{lstlisting}[firstnumber=last]
    create => sub {
        my ($app, %param) = @_;
        my $person_id = $param{person_id};
    }, # POST /person
);
\end{lstlisting}
\end{itemize}
\end{frame}

%%%%%%%%%%%%%%%%%%%%%%%%%%%%%%%%%%%%%%%%%%%%%%%%%%%%%%%%%%%%%%%%%%%%%%%%%%%%%%%%%%%%%%%%%%%%%%%%%%%

\begin{frame}{Formats}

    \begin{itemize}
        \item Format wird wie eine Dateiendung definiert
        \pause
        \item \texttt{/resource/:id.:format}
        \pause
        \begin{itemize}
            \item YAML: \texttt{/resource/123.yml}
            \pause
            \item JSON: \texttt{/resource/123.json}
            \pause
            \item Dumper: \texttt{/resource/123.dump}
            \pause
        \end{itemize}
        \item Eingabeformat entspricht Ausgabeformat
    \end{itemize}

\end{frame}

%%%%%%%%%%%%%%%%%%%%%%%%%%%%%%%%%%%%%%%%%%%%%%%%%%%%%%%%%%%%%%%%%%%%%%%%%%%%%%%%%%%%%%%%%%%%%%%%%%%

\begin{frame}[fragile]{Chaining}

\begin{itemize}
\item Verketten
\pause
\item[]
\begin{lstlisting}
resource("foo",
    single_id => sub {
\end{lstlisting}\pause\begin{lstlisting}[firstnumber=last]
        resource("bar",
            read => sub {
\end{lstlisting}\pause\begin{lstlisting}[firstnumber=last]
                my ($app, %param) = @_;
                my $foo_id = $param{foo_id};
                my $bar_id = $param{bar_id};
\end{lstlisting}\pause\begin{lstlisting}[firstnumber=last]
            } # GET /foo/123/bar/456
        );
    },
);
\end{lstlisting}
\end{itemize}
\end{frame}

%%%%%%%%%%%%%%%%%%%%%%%%%%%%%%%%%%%%%%%%%%%%%%%%%%%%%%%%%%%%%%%%%%%%%%%%%%%%%%%%%%%%%%%%%%%%%%%%%%%

\begin{frame}[fragile]{Validation}

\begin{itemize}
\item \texttt{JSON::Schema}
\pause
\item[]
\begin{lstlisting}
resource("person",
    create => sub :RequestSchema(
\end{lstlisting}\pause\begin{lstlisting}[firstnumber=last]
        type => 'object',
        properties => {
            firstName => { type => 'string' },
            lastName  => { type => 'string' },
            age => {
                type => 'integer',
                minimum => 0
            }
        }
\end{lstlisting}\pause\begin{lstlisting}[firstnumber=last]
    ) {
        my $app = shift;
    },
);
\end{lstlisting}
\end{itemize}
\end{frame}

%%%%%%%%%%%%%%%%%%%%%%%%%%%%%%%%%%%%%%%%%%%%%%%%%%%%%%%%%%%%%%%%%%%%%%%%%%%%%%%%%%%%%%%%%%%%%%%%%%%

\begin{frame}[fragile]{Documentation}

\begin{itemize}
\item Dokumentieren
\pause
\item[]
\begin{lstlisting}
resource("person",
    index => sub :Description(List all persons in this room) {
        my $app = shift;
    },
);
\end{lstlisting}\pause\begin{lstlisting}[firstnumber=last]

publish_apiblueprint("/doc");
\end{lstlisting}
\end{itemize}
\end{frame}

%%%%%%%%%%%%%%%%%%%%%%%%%%%%%%%%%%%%%%%%%%%%%%%%%%%%%%%%%%%%%%%%%%%%%%%%%%%%%%%%%%%%%%%%%%%%%%%%%%%

\begin{frame}[fragile]{Documentation}

\begin{itemize}

\item API Blueprint\cite{apiblueprint}
\pause
\item Markdown Syntax
\pause

\item
{\tiny
\begin{multicols}{3}
\begin{verbatim}
FORMAT: 1A
# Group person
## Index [/person.{format}]
### Index [GET]
List all persons in this room
+ Parameters
    + format (required, string)
        + Values
            + `json`
            + `yaml`
            + `dump`
+ Request JSON (application/json)
    + `/person.json`
    + Body
            {
               "limit" : 10
            }
+ Response 200 (application/json)
    + Body
            [
               {
                  "name" : "Alice"
               },
               {
                  "name" : "Bob"
               }
            ]
\end{verbatim}\columnbreak\begin{verbatim}
+ Request YAML (text/x-yaml)
    + `/person.yml`
    + Body
            --- 
            limit: 10
+ Response 200 (text/x-yaml)
    + Body
            --- 
            - 
              name: Alice
            - 
              name: Bob
\end{verbatim}\columnbreak\begin{verbatim}
+ Request DUMP (text/x-perl)
    + `/person.dump`
    + Body
            {
              'limit' => 10
            }
+ Response 200 (text/x-perl)
    + Body
            [
              {
                'name' => 'Alice'
              },
              {
                'name' => 'Bob'
              }
            ]
\end{verbatim}
\end{multicols}
}

\pause
\item Und mit aglio\cite{aglio} wird es auch schön :-)
\end{itemize}

\end{frame}

%%%%%%%%%%%%%%%%%%%%%%%%%%%%%%%%%%%%%%%%%%%%%%%%%%%%%%%%%%%%%%%%%%%%%%%%%%%%%%%%%%%%%%%%%%%%%%%%%%%

\begin{frame}[fragile]{Testing}

\begin{itemize}
\item Testen mit Dumper
\pause
\item[]
\begin{lstlisting}
use Dancer2::Plugin::CRUD::Test;
\end{lstlisting}\pause\begin{lstlisting}[firstnumber=last]
my $T = Dancer2::Plugin::CRUD::Test->new('Webservice');
\end{lstlisting}\pause\begin{lstlisting}[firstnumber=last]
my $R = $T->action(
\end{lstlisting}\pause\begin{lstlisting}[firstnumber=last]
    'create',
\end{lstlisting}\pause\begin{lstlisting}[firstnumber=last]
    '/echo',
\end{lstlisting}\pause\begin{lstlisting}[firstnumber=last]
    { 'xx' => 'yy' },
\end{lstlisting}\pause\begin{lstlisting}[firstnumber=last]
    expect => 200
\end{lstlisting}\pause\begin{lstlisting}[firstnumber=last]
);
my $data = $R->data;
\end{lstlisting}\pause\begin{lstlisting}[firstnumber=last]
is_deeply($data, { 'xx' => 'yy' } );
\end{lstlisting}
\end{itemize}
\end{frame}


%%%%%%%%%%%%%%%%%%%%%%%%%%%%%%%%%%%%%%%%%%%%%%%%%%%%%%%%%%%%%%%%%%%%%%%%%%%%%%%%%%%%%%%%%%%%%%%%%%%

\appendix

%%%%%%%%%%%%%%%%%%%%%%%%%%%%%%%%%%%%%%%%%%%%%%%%%%%%%%%%%%%%%%%%%%%%%%%%%%%%%%%%%%%%%%%%%%%%%%%%%%%

\section<presentation>*{\appendixname}

%%%%%%%%%%%%%%%%%%%%%%%%%%%%%%%%%%%%%%%%%%%%%%%%%%%%%%%%%%%%%%%%%%%%%%%%%%%%%%%%%%%%%%%%%%%%%%%%%%%

\subsection<presentation>*{Weiterführende Literatur}

%%%%%%%%%%%%%%%%%%%%%%%%%%%%%%%%%%%%%%%%%%%%%%%%%%%%%%%%%%%%%%%%%%%%%%%%%%%%%%%%%%%%%%%%%%%%%%%%%%%

\begin{frame}[allowframebreaks]
  \frametitle<presentation>{Weiterführende Literatur}

\bibliography{talk-gpw2015-de}
\bibliographystyle{unsrt}

\end{frame}

%%%%%%%%%%%%%%%%%%%%%%%%%%%%%%%%%%%%%%%%%%%%%%%%%%%%%%%%%%%%%%%%%%%%%%%%%%%%%%%%%%%%%%%%%%%%%%%%%%%

\end{document}

